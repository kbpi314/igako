\documentclass[10pt,notitlepage,onecolumn,aps,pra]{revtex4-1}


    
\usepackage[T1]{fontenc}
\usepackage{graphicx}
% We will generate all images so they have a width \maxwidth. This means
% that they will get their normal width if they fit onto the page, but
% are scaled down if they would overflow the margins.
\makeatletter
\def\maxwidth{\ifdim\Gin@nat@width>\linewidth\linewidth
\else\Gin@nat@width\fi}
\makeatother
\let\Oldincludegraphics\includegraphics
% Set max figure width to be 80% of text width, for now hardcoded.
\renewcommand{\includegraphics}[1]{\Oldincludegraphics[width=.8\maxwidth]{#1}}
% Ensure that by default, figures have no caption (until we provide a
% proper Figure object with a Caption API and a way to capture that
% in the conversion process - todo).
\usepackage{caption}
\DeclareCaptionLabelFormat{nolabel}{}
\captionsetup{labelformat=nolabel}

\usepackage{adjustbox} % Used to constrain images to a maximum size
\usepackage{xcolor} % Allow colors to be defined
\usepackage{enumerate} % Needed for markdown enumerations to work
\usepackage{geometry} % Used to adjust the document margins
\usepackage{amsmath} % Equations
\usepackage{amssymb} % Equations
\usepackage{textcomp} % defines textquotesingle
% Hack from http://tex.stackexchange.com/a/47451/13684:
\AtBeginDocument{%
    \def\PYZsq{\textquotesingle}% Upright quotes in Pygmentized code
}
\usepackage{upquote} % Upright quotes for verbatim code
\usepackage{eurosym} % defines \euro
\usepackage[mathletters]{ucs} % Extended unicode (utf-8) support
\usepackage[utf8x]{inputenc} % Allow utf-8 characters in the tex document
\usepackage{fancyvrb} % verbatim replacement that allows latex
\usepackage{grffile} % extends the file name processing of package graphics
                     % to support a larger range
% The hyperref package gives us a pdf with properly built
% internal navigation ('pdf bookmarks' for the table of contents,
% internal cross-reference links, web links for URLs, etc.)
\usepackage{hyperref}
\usepackage{booktabs}  % table support for pandoc > 1.12.2
\usepackage[inline]{enumitem} % IRkernel/repr support (it uses the enumerate* environment)
\usepackage[normalem]{ulem} % ulem is needed to support strikethroughs (\sout)
                            % normalem makes italics be italics, not underlines
\usepackage{braket}

\usepackage{titlesec}
\usepackage{alphalph}
\usepackage{pgffor}
\renewcommand*{\thesubsection}{
    \alphalph{\value{subsection}}
}
\newcommand\crule[3][black]{\textcolor{#1}{\rule{#2}{#3}}}
\def\otuPhylumHousing{k\_\_Bacteria;p\_\_Bacteroidetes/39.82,k\_\_Bacteria;p\_\_Firmicutes/66.34}
\def\otuPhylumHousing{k\_\_Bacteria;p\_\_Bacteroidetes/29.82,k\_\_Bacteria;p\_\_Firmicutes/59.48,k\_\_Bacteria;p\_\_Proteobacteria/11.78}
\definecolor{k\_\_Bacteria;p\_\_Bacteroidetes}{RGB}{82,175,67}
\definecolor{k\_\_Bacteria;p\_\_Firmicutes}{RGB}{240,108,69}
\definecolor{k\_\_Bacteria;p\_\_Proteobacteria}{RGB}{178,148,199}
\def\otuPhylum{k\_\_Bacteria;p\_\_Bacteroidetes/33.03,k\_\_Bacteria;p\_\_Firmicutes/62.27,k\_\_Bacteria;p\_\_Proteobacteria/11.78}
\def\otuPhylumStrain{k\_\_Bacteria;p\_\_Bacteroidetes/34.87,k\_\_Bacteria;p\_\_Firmicutes/63.13,k\_\_Bacteria;p\_\_Proteobacteria/11.67}
\def\otuPhylumStrain{k\_\_Bacteria;p\_\_Bacteroidetes/31.44,k\_\_Bacteria;p\_\_Firmicutes/61.40,k\_\_Bacteria;p\_\_Proteobacteria/11.86}
\definecolor{k\_\_Bacteria;p\_\_Bacteroidetes}{RGB}{82,175,67}
\definecolor{k\_\_Bacteria;p\_\_Firmicutes}{RGB}{240,108,69}
\definecolor{k\_\_Bacteria;p\_\_Proteobacteria}{RGB}{178,148,199}
\def\otuPhylum{k\_\_Bacteria;p\_\_Bacteroidetes/33.03,k\_\_Bacteria;p\_\_Firmicutes/62.27,k\_\_Bacteria;p\_\_Proteobacteria/11.78}
\def\otuGenusHousing{k\_\_Bacteria;p\_\_Bacteroidetes;c\_\_Bacteroidia;o\_\_Bacteroidales;f\_\_S24-7;g\_\_/39.77,k\_\_Bacteria;p\_\_Firmicutes;c\_\_Bacilli;o\_\_Lactobacillales;f\_\_Lactobacillaceae;g\_\_Lactobacillus/38.31,k\_\_Bacteria;p\_\_Firmicutes;c\_\_Bacilli;o\_\_Turicibacterales;f\_\_Turicibacteraceae;g\_\_Turicibacter/13.32,k\_\_Bacteria;p\_\_Firmicutes;c\_\_Clostridia;o\_\_Clostridiales;f\_\_Clostridiaceae;g\_\_Candidatus Arthromitus/30.05}
\def\otuGenusHousing{k\_\_Bacteria;p\_\_Bacteroidetes;c\_\_Bacteroidia;o\_\_Bacteroidales;f\_\_S24-7;g\_\_/29.78,k\_\_Bacteria;p\_\_Firmicutes;c\_\_Bacilli;o\_\_Lactobacillales;f\_\_Lactobacillaceae;g\_\_Lactobacillus/42.35,k\_\_Bacteria;p\_\_Firmicutes;c\_\_Bacilli;o\_\_Turicibacterales;f\_\_Turicibacteraceae;g\_\_Turicibacter/11.13,k\_\_Bacteria;p\_\_Firmicutes;c\_\_Clostridia;o\_\_Clostridiales;f\_\_Clostridiaceae;g\_\_Candidatus Arthromitus/38.47,k\_\_Bacteria;p\_\_Proteobacteria;c\_\_Deltaproteobacteria;o\_\_Desulfovibrionales;f\_\_Desulfovibrionaceae;g\_\_Desulfovibrio/11.94}
\definecolor{k\_\_Bacteria;p\_\_Bacteroidetes;c\_\_Bacteroidia;o\_\_Bacteroidales;f\_\_S24-7;g\_\_}{RGB}{172,220,133}
\definecolor{k\_\_Bacteria;p\_\_Firmicutes;c\_\_Bacilli;o\_\_Lactobacillales;f\_\_Lactobacillaceae;g\_\_Lactobacillus}{RGB}{173,156,110}
\definecolor{k\_\_Bacteria;p\_\_Firmicutes;c\_\_Bacilli;o\_\_Turicibacterales;f\_\_Turicibacteraceae;g\_\_Turicibacter}{RGB}{236,154,145}
\definecolor{k\_\_Bacteria;p\_\_Firmicutes;c\_\_Clostridia;o\_\_Clostridiales;f\_\_Clostridiaceae;g\_\_Candidatus Arthromitus}{RGB}{233,62,63}
\definecolor{k\_\_Bacteria;p\_\_Proteobacteria;c\_\_Deltaproteobacteria;o\_\_Desulfovibrionales;f\_\_Desulfovibrionaceae;g\_\_Desulfovibrio}{RGB}{193,174,153}
\def\otuGenus{k\_\_Bacteria;p\_\_Bacteroidetes;c\_\_Bacteroidia;o\_\_Bacteroidales;f\_\_S24-7;g\_\_/32.99,k\_\_Bacteria;p\_\_Firmicutes;c\_\_Bacilli;o\_\_Lactobacillales;f\_\_Lactobacillaceae;g\_\_Lactobacillus/40.60,k\_\_Bacteria;p\_\_Firmicutes;c\_\_Bacilli;o\_\_Turicibacterales;f\_\_Turicibacteraceae;g\_\_Turicibacter/11.86,k\_\_Bacteria;p\_\_Firmicutes;c\_\_Clostridia;o\_\_Clostridiales;f\_\_Clostridiaceae;g\_\_Candidatus Arthromitus/33.42,k\_\_Bacteria;p\_\_Proteobacteria;c\_\_Deltaproteobacteria;o\_\_Desulfovibrionales;f\_\_Desulfovibrionaceae;g\_\_Desulfovibrio/11.94}
\def\otuGenusStrain{k\_\_Bacteria;p\_\_Bacteroidetes;c\_\_Bacteroidia;o\_\_Bacteroidales;f\_\_S24-7;g\_\_/34.84,k\_\_Bacteria;p\_\_Firmicutes;c\_\_Bacilli;o\_\_Lactobacillales;f\_\_Lactobacillaceae;g\_\_Lactobacillus/38.57,k\_\_Bacteria;p\_\_Firmicutes;c\_\_Clostridia;o\_\_Clostridiales;f\_\_Clostridiaceae;g\_\_Candidatus Arthromitus/36.53,k\_\_Bacteria;p\_\_Proteobacteria;c\_\_Deltaproteobacteria;o\_\_Desulfovibrionales;f\_\_Desulfovibrionaceae;g\_\_Desulfovibrio/11.34}
\def\otuGenusStrain{k\_\_Bacteria;p\_\_Bacteroidetes;c\_\_Bacteroidia;o\_\_Bacteroidales;f\_\_S24-7;g\_\_/31.39,k\_\_Bacteria;p\_\_Firmicutes;c\_\_Bacilli;o\_\_Lactobacillales;f\_\_Lactobacillaceae;g\_\_Lactobacillus/42.37,k\_\_Bacteria;p\_\_Firmicutes;c\_\_Bacilli;o\_\_Turicibacterales;f\_\_Turicibacteraceae;g\_\_Turicibacter/11.86,k\_\_Bacteria;p\_\_Firmicutes;c\_\_Clostridia;o\_\_Clostridiales;f\_\_Clostridiaceae;g\_\_Candidatus Arthromitus/27.18,k\_\_Bacteria;p\_\_Proteobacteria;c\_\_Deltaproteobacteria;o\_\_Desulfovibrionales;f\_\_Desulfovibrionaceae;g\_\_Desulfovibrio/12.24}
\definecolor{k\_\_Bacteria;p\_\_Bacteroidetes;c\_\_Bacteroidia;o\_\_Bacteroidales;f\_\_S24-7;g\_\_}{RGB}{172,220,133}
\definecolor{k\_\_Bacteria;p\_\_Firmicutes;c\_\_Bacilli;o\_\_Lactobacillales;f\_\_Lactobacillaceae;g\_\_Lactobacillus}{RGB}{173,156,110}
\definecolor{k\_\_Bacteria;p\_\_Firmicutes;c\_\_Bacilli;o\_\_Turicibacterales;f\_\_Turicibacteraceae;g\_\_Turicibacter}{RGB}{236,154,145}
\definecolor{k\_\_Bacteria;p\_\_Firmicutes;c\_\_Clostridia;o\_\_Clostridiales;f\_\_Clostridiaceae;g\_\_Candidatus Arthromitus}{RGB}{233,62,63}
\definecolor{k\_\_Bacteria;p\_\_Proteobacteria;c\_\_Deltaproteobacteria;o\_\_Desulfovibrionales;f\_\_Desulfovibrionaceae;g\_\_Desulfovibrio}{RGB}{193,174,153}
\def\otuGenus{k\_\_Bacteria;p\_\_Bacteroidetes;c\_\_Bacteroidia;o\_\_Bacteroidales;f\_\_S24-7;g\_\_/32.99,k\_\_Bacteria;p\_\_Firmicutes;c\_\_Bacilli;o\_\_Lactobacillales;f\_\_Lactobacillaceae;g\_\_Lactobacillus/40.60,k\_\_Bacteria;p\_\_Firmicutes;c\_\_Bacilli;o\_\_Turicibacterales;f\_\_Turicibacteraceae;g\_\_Turicibacter/11.86,k\_\_Bacteria;p\_\_Firmicutes;c\_\_Clostridia;o\_\_Clostridiales;f\_\_Clostridiaceae;g\_\_Candidatus Arthromitus/33.42,k\_\_Bacteria;p\_\_Proteobacteria;c\_\_Deltaproteobacteria;o\_\_Desulfovibrionales;f\_\_Desulfovibrionaceae;g\_\_Desulfovibrio/11.94}
\definecolor{color0}{RGB}{228,26,28}
\definecolor{color1}{RGB}{55,126,184}
\definecolor{color2}{RGB}{77,175,74}
\definecolor{color3}{RGB}{152,78,163}
\definecolor{color4}{RGB}{255,127,0}
\definecolor{color5}{RGB}{255,255,51}
\definecolor{color6}{RGB}{166,86,40}
\def\Housing{color0/Littermate,color1/non-co-housed}
\def\Strain{color3/IgAKO,color2/WT}
\def\otuPhylumHousing{k\_\_Bacteria;p\_\_Bacteroidetes/39.82,k\_\_Bacteria;p\_\_Firmicutes/66.34}
\def\otuPhylumHousing{k\_\_Bacteria;p\_\_Bacteroidetes/29.82,k\_\_Bacteria;p\_\_Firmicutes/59.48,k\_\_Bacteria;p\_\_Proteobacteria/11.78}
\definecolor{k\_\_Bacteria;p\_\_Bacteroidetes}{RGB}{82,175,67}
\definecolor{k\_\_Bacteria;p\_\_Firmicutes}{RGB}{240,108,69}
\definecolor{k\_\_Bacteria;p\_\_Proteobacteria}{RGB}{178,148,199}
\def\otuPhylum{k\_\_Bacteria;p\_\_Bacteroidetes/33.03,k\_\_Bacteria;p\_\_Firmicutes/62.27,k\_\_Bacteria;p\_\_Proteobacteria/11.78}
\def\otuPhylumStrain{k\_\_Bacteria;p\_\_Bacteroidetes/34.87,k\_\_Bacteria;p\_\_Firmicutes/63.13,k\_\_Bacteria;p\_\_Proteobacteria/11.67}
\def\otuPhylumStrain{k\_\_Bacteria;p\_\_Bacteroidetes/31.44,k\_\_Bacteria;p\_\_Firmicutes/61.40,k\_\_Bacteria;p\_\_Proteobacteria/11.86}
\definecolor{k\_\_Bacteria;p\_\_Bacteroidetes}{RGB}{82,175,67}
\definecolor{k\_\_Bacteria;p\_\_Firmicutes}{RGB}{240,108,69}
\definecolor{k\_\_Bacteria;p\_\_Proteobacteria}{RGB}{178,148,199}
\def\otuPhylum{k\_\_Bacteria;p\_\_Bacteroidetes/33.03,k\_\_Bacteria;p\_\_Firmicutes/62.27,k\_\_Bacteria;p\_\_Proteobacteria/11.78}
\def\otuGenusHousing{k\_\_Bacteria;p\_\_Bacteroidetes;c\_\_Bacteroidia;o\_\_Bacteroidales;f\_\_S24-7;g\_\_/39.77,k\_\_Bacteria;p\_\_Firmicutes;c\_\_Bacilli;o\_\_Lactobacillales;f\_\_Lactobacillaceae;g\_\_Lactobacillus/38.31,k\_\_Bacteria;p\_\_Firmicutes;c\_\_Bacilli;o\_\_Turicibacterales;f\_\_Turicibacteraceae;g\_\_Turicibacter/13.32,k\_\_Bacteria;p\_\_Firmicutes;c\_\_Clostridia;o\_\_Clostridiales;f\_\_Clostridiaceae;g\_\_Candidatus Arthromitus/30.05}
\def\otuGenusHousing{k\_\_Bacteria;p\_\_Bacteroidetes;c\_\_Bacteroidia;o\_\_Bacteroidales;f\_\_S24-7;g\_\_/29.78,k\_\_Bacteria;p\_\_Firmicutes;c\_\_Bacilli;o\_\_Lactobacillales;f\_\_Lactobacillaceae;g\_\_Lactobacillus/42.35,k\_\_Bacteria;p\_\_Firmicutes;c\_\_Bacilli;o\_\_Turicibacterales;f\_\_Turicibacteraceae;g\_\_Turicibacter/11.13,k\_\_Bacteria;p\_\_Firmicutes;c\_\_Clostridia;o\_\_Clostridiales;f\_\_Clostridiaceae;g\_\_Candidatus Arthromitus/38.47,k\_\_Bacteria;p\_\_Proteobacteria;c\_\_Deltaproteobacteria;o\_\_Desulfovibrionales;f\_\_Desulfovibrionaceae;g\_\_Desulfovibrio/11.94}
\definecolor{k\_\_Bacteria;p\_\_Bacteroidetes;c\_\_Bacteroidia;o\_\_Bacteroidales;f\_\_S24-7;g\_\_}{RGB}{172,220,133}
\definecolor{k\_\_Bacteria;p\_\_Firmicutes;c\_\_Bacilli;o\_\_Lactobacillales;f\_\_Lactobacillaceae;g\_\_Lactobacillus}{RGB}{173,156,110}
\definecolor{k\_\_Bacteria;p\_\_Firmicutes;c\_\_Bacilli;o\_\_Turicibacterales;f\_\_Turicibacteraceae;g\_\_Turicibacter}{RGB}{236,154,145}
\definecolor{k\_\_Bacteria;p\_\_Firmicutes;c\_\_Clostridia;o\_\_Clostridiales;f\_\_Clostridiaceae;g\_\_Candidatus Arthromitus}{RGB}{233,62,63}
\definecolor{k\_\_Bacteria;p\_\_Proteobacteria;c\_\_Deltaproteobacteria;o\_\_Desulfovibrionales;f\_\_Desulfovibrionaceae;g\_\_Desulfovibrio}{RGB}{193,174,153}
\def\otuGenus{k\_\_Bacteria;p\_\_Bacteroidetes;c\_\_Bacteroidia;o\_\_Bacteroidales;f\_\_S24-7;g\_\_/32.99,k\_\_Bacteria;p\_\_Firmicutes;c\_\_Bacilli;o\_\_Lactobacillales;f\_\_Lactobacillaceae;g\_\_Lactobacillus/40.60,k\_\_Bacteria;p\_\_Firmicutes;c\_\_Bacilli;o\_\_Turicibacterales;f\_\_Turicibacteraceae;g\_\_Turicibacter/11.86,k\_\_Bacteria;p\_\_Firmicutes;c\_\_Clostridia;o\_\_Clostridiales;f\_\_Clostridiaceae;g\_\_Candidatus Arthromitus/33.42,k\_\_Bacteria;p\_\_Proteobacteria;c\_\_Deltaproteobacteria;o\_\_Desulfovibrionales;f\_\_Desulfovibrionaceae;g\_\_Desulfovibrio/11.94}
\def\otuGenusStrain{k\_\_Bacteria;p\_\_Bacteroidetes;c\_\_Bacteroidia;o\_\_Bacteroidales;f\_\_S24-7;g\_\_/34.84,k\_\_Bacteria;p\_\_Firmicutes;c\_\_Bacilli;o\_\_Lactobacillales;f\_\_Lactobacillaceae;g\_\_Lactobacillus/38.57,k\_\_Bacteria;p\_\_Firmicutes;c\_\_Clostridia;o\_\_Clostridiales;f\_\_Clostridiaceae;g\_\_Candidatus Arthromitus/36.53,k\_\_Bacteria;p\_\_Proteobacteria;c\_\_Deltaproteobacteria;o\_\_Desulfovibrionales;f\_\_Desulfovibrionaceae;g\_\_Desulfovibrio/11.34}
\def\otuGenusStrain{k\_\_Bacteria;p\_\_Bacteroidetes;c\_\_Bacteroidia;o\_\_Bacteroidales;f\_\_S24-7;g\_\_/31.39,k\_\_Bacteria;p\_\_Firmicutes;c\_\_Bacilli;o\_\_Lactobacillales;f\_\_Lactobacillaceae;g\_\_Lactobacillus/42.37,k\_\_Bacteria;p\_\_Firmicutes;c\_\_Bacilli;o\_\_Turicibacterales;f\_\_Turicibacteraceae;g\_\_Turicibacter/11.86,k\_\_Bacteria;p\_\_Firmicutes;c\_\_Clostridia;o\_\_Clostridiales;f\_\_Clostridiaceae;g\_\_Candidatus Arthromitus/27.18,k\_\_Bacteria;p\_\_Proteobacteria;c\_\_Deltaproteobacteria;o\_\_Desulfovibrionales;f\_\_Desulfovibrionaceae;g\_\_Desulfovibrio/12.24}
\definecolor{k\_\_Bacteria;p\_\_Bacteroidetes;c\_\_Bacteroidia;o\_\_Bacteroidales;f\_\_S24-7;g\_\_}{RGB}{172,220,133}
\definecolor{k\_\_Bacteria;p\_\_Firmicutes;c\_\_Bacilli;o\_\_Lactobacillales;f\_\_Lactobacillaceae;g\_\_Lactobacillus}{RGB}{173,156,110}
\definecolor{k\_\_Bacteria;p\_\_Firmicutes;c\_\_Bacilli;o\_\_Turicibacterales;f\_\_Turicibacteraceae;g\_\_Turicibacter}{RGB}{236,154,145}
\definecolor{k\_\_Bacteria;p\_\_Firmicutes;c\_\_Clostridia;o\_\_Clostridiales;f\_\_Clostridiaceae;g\_\_Candidatus Arthromitus}{RGB}{233,62,63}
\definecolor{k\_\_Bacteria;p\_\_Proteobacteria;c\_\_Deltaproteobacteria;o\_\_Desulfovibrionales;f\_\_Desulfovibrionaceae;g\_\_Desulfovibrio}{RGB}{193,174,153}
\def\otuGenus{k\_\_Bacteria;p\_\_Bacteroidetes;c\_\_Bacteroidia;o\_\_Bacteroidales;f\_\_S24-7;g\_\_/32.99,k\_\_Bacteria;p\_\_Firmicutes;c\_\_Bacilli;o\_\_Lactobacillales;f\_\_Lactobacillaceae;g\_\_Lactobacillus/40.60,k\_\_Bacteria;p\_\_Firmicutes;c\_\_Bacilli;o\_\_Turicibacterales;f\_\_Turicibacteraceae;g\_\_Turicibacter/11.86,k\_\_Bacteria;p\_\_Firmicutes;c\_\_Clostridia;o\_\_Clostridiales;f\_\_Clostridiaceae;g\_\_Candidatus Arthromitus/33.42,k\_\_Bacteria;p\_\_Proteobacteria;c\_\_Deltaproteobacteria;o\_\_Desulfovibrionales;f\_\_Desulfovibrionaceae;g\_\_Desulfovibrio/11.94}
\definecolor{color0}{RGB}{228,26,28}
\definecolor{color1}{RGB}{55,126,184}
\definecolor{color2}{RGB}{77,175,74}
\definecolor{color3}{RGB}{152,78,163}
\definecolor{color4}{RGB}{255,127,0}
\definecolor{color5}{RGB}{255,255,51}
\definecolor{color6}{RGB}{166,86,40}
\def\Housing{color0/Littermate,color1/non-co-housed}
\def\Strain{color3/IgAKO,color2/WT}


    
    % Colors for the hyperref package
    \definecolor{urlcolor}{rgb}{0,.145,.698}
    \definecolor{linkcolor}{rgb}{.71,0.21,0.01}
    \definecolor{citecolor}{rgb}{.12,.54,.11}

    % ANSI colors
    \definecolor{ansi-black}{HTML}{3E424D}
    \definecolor{ansi-black-intense}{HTML}{282C36}
    \definecolor{ansi-red}{HTML}{E75C58}
    \definecolor{ansi-red-intense}{HTML}{B22B31}
    \definecolor{ansi-green}{HTML}{00A250}
    \definecolor{ansi-green-intense}{HTML}{007427}
    \definecolor{ansi-yellow}{HTML}{DDB62B}
    \definecolor{ansi-yellow-intense}{HTML}{B27D12}
    \definecolor{ansi-blue}{HTML}{208FFB}
    \definecolor{ansi-blue-intense}{HTML}{0065CA}
    \definecolor{ansi-magenta}{HTML}{D160C4}
    \definecolor{ansi-magenta-intense}{HTML}{A03196}
    \definecolor{ansi-cyan}{HTML}{60C6C8}
    \definecolor{ansi-cyan-intense}{HTML}{258F8F}
    \definecolor{ansi-white}{HTML}{C5C1B4}
    \definecolor{ansi-white-intense}{HTML}{A1A6B2}
    \definecolor{ansi-default-inverse-fg}{HTML}{FFFFFF}
    \definecolor{ansi-default-inverse-bg}{HTML}{000000}

    % commands and environments needed by pandoc snippets
    % extracted from the output of `pandoc -s`
    \providecommand{\tightlist}{%
      \setlength{\itemsep}{0pt}\setlength{\parskip}{0pt}}
    \DefineVerbatimEnvironment{Highlighting}{Verbatim}{commandchars=\\\{\}}
    % Add ',fontsize=\small' for more characters per line
    \newenvironment{Shaded}{}{}
    \newcommand{\KeywordTok}[1]{\textcolor[rgb]{0.00,0.44,0.13}{\textbf{{#1}}}}
    \newcommand{\DataTypeTok}[1]{\textcolor[rgb]{0.56,0.13,0.00}{{#1}}}
    \newcommand{\DecValTok}[1]{\textcolor[rgb]{0.25,0.63,0.44}{{#1}}}
    \newcommand{\BaseNTok}[1]{\textcolor[rgb]{0.25,0.63,0.44}{{#1}}}
    \newcommand{\FloatTok}[1]{\textcolor[rgb]{0.25,0.63,0.44}{{#1}}}
    \newcommand{\CharTok}[1]{\textcolor[rgb]{0.25,0.44,0.63}{{#1}}}
    \newcommand{\StringTok}[1]{\textcolor[rgb]{0.25,0.44,0.63}{{#1}}}
    \newcommand{\CommentTok}[1]{\textcolor[rgb]{0.38,0.63,0.69}{\textit{{#1}}}}
    \newcommand{\OtherTok}[1]{\textcolor[rgb]{0.00,0.44,0.13}{{#1}}}
    \newcommand{\AlertTok}[1]{\textcolor[rgb]{1.00,0.00,0.00}{\textbf{{#1}}}}
    \newcommand{\FunctionTok}[1]{\textcolor[rgb]{0.02,0.16,0.49}{{#1}}}
    \newcommand{\RegionMarkerTok}[1]{{#1}}
    \newcommand{\ErrorTok}[1]{\textcolor[rgb]{1.00,0.00,0.00}{\textbf{{#1}}}}
    \newcommand{\NormalTok}[1]{{#1}}
    
    % Additional commands for more recent versions of Pandoc
    \newcommand{\ConstantTok}[1]{\textcolor[rgb]{0.53,0.00,0.00}{{#1}}}
    \newcommand{\SpecialCharTok}[1]{\textcolor[rgb]{0.25,0.44,0.63}{{#1}}}
    \newcommand{\VerbatimStringTok}[1]{\textcolor[rgb]{0.25,0.44,0.63}{{#1}}}
    \newcommand{\SpecialStringTok}[1]{\textcolor[rgb]{0.73,0.40,0.53}{{#1}}}
    \newcommand{\ImportTok}[1]{{#1}}
    \newcommand{\DocumentationTok}[1]{\textcolor[rgb]{0.73,0.13,0.13}{\textit{{#1}}}}
    \newcommand{\AnnotationTok}[1]{\textcolor[rgb]{0.38,0.63,0.69}{\textbf{\textit{{#1}}}}}
    \newcommand{\CommentVarTok}[1]{\textcolor[rgb]{0.38,0.63,0.69}{\textbf{\textit{{#1}}}}}
    \newcommand{\VariableTok}[1]{\textcolor[rgb]{0.10,0.09,0.49}{{#1}}}
    \newcommand{\ControlFlowTok}[1]{\textcolor[rgb]{0.00,0.44,0.13}{\textbf{{#1}}}}
    \newcommand{\OperatorTok}[1]{\textcolor[rgb]{0.40,0.40,0.40}{{#1}}}
    \newcommand{\BuiltInTok}[1]{{#1}}
    \newcommand{\ExtensionTok}[1]{{#1}}
    \newcommand{\PreprocessorTok}[1]{\textcolor[rgb]{0.74,0.48,0.00}{{#1}}}
    \newcommand{\AttributeTok}[1]{\textcolor[rgb]{0.49,0.56,0.16}{{#1}}}
    \newcommand{\InformationTok}[1]{\textcolor[rgb]{0.38,0.63,0.69}{\textbf{\textit{{#1}}}}}
    \newcommand{\WarningTok}[1]{\textcolor[rgb]{0.38,0.63,0.69}{\textbf{\textit{{#1}}}}}
    
    
    % Define a nice break command that doesn't care if a line doesn't already
    % exist.
    \def\br{\hspace*{\fill} \\* }
    % Math Jax compatibility definitions
    \def\gt{>}
    \def\lt{<}
    \let\Oldtex\TeX
    \let\Oldlatex\LaTeX
    \renewcommand{\TeX}{\textrm{\Oldtex}}
    \renewcommand{\LaTeX}{\textrm{\Oldlatex}}
    % Document parameters
    % Document title
    
    
    
    

    
    % Prevent overflowing lines due to hard-to-break entities
    \sloppy 
    % Setup hyperref package
    \hypersetup{
      breaklinks=true,  % so long urls are correctly broken across lines
      colorlinks=true,
      urlcolor=urlcolor,
      linkcolor=linkcolor,
      citecolor=citecolor,
      }
    % Slightly bigger margins than the latex defaults
    
    \geometry{verbose,tmargin=1in,bmargin=1in,lmargin=1in,rmargin=1in}
    
    

\begin{document}
    
    \title{MMEDS Analysis Summary}\author{Clemente Lab}\affiliation{Icahn School of Medicine at Mount Sinai}

\date{\today}
\maketitle


    
    

    
    \hypertarget{table-statistics-summary}{%
\section{Table Statistics Summary}\label{table-statistics-summary}}

    \hypertarget{dada2-statistics}{%
\subsection{Dada2 Statistics}\label{dada2-statistics}}

    
    \begin{center}
    \adjustimage{max size={0.9\linewidth}{0.9\paperheight}}{mkstapylton@gmail.com-mattS-qiime2_files/mkstapylton@gmail.com-mattS-qiime2_7_0.png}
    \end{center}
    { \hspace*{\fill} \\}
    

    The above plot represents number of input reads (total bar length) and
retained (magenta) after quality control filtering, including denoising
and chimera checking.

    \pagebreak

    \hypertarget{taxonomy-summary}{%
\section{Taxonomy Summary}\label{taxonomy-summary}}

    \hypertarget{interpreting-taxonomy-results}{%
\subsection{Interpreting Taxonomy
Results}\label{interpreting-taxonomy-results}}

    Taxonomy plots represent the abundance of different taxa using stacked
plots on a per-sample or per-group (averaged) basis. Data is normalized
so that abundances per sample or per group add up to 100\%. When using
group-based taxonomy plots, it should be noted that only average
abundances are shown per group and taxa: this can induce visual biases
when a small number of samples in a group have significantly higher
abundance of a given taxa compared to the rest of samples in the group,
and give the (incorrect) impression that the group as a whole has high
high abundance of the taxa.

    \hypertarget{phylum-level}{%
\subsection{Phylum Level}\label{phylum-level}}

    
    \begin{center}
    \adjustimage{max size={0.9\linewidth}{0.9\paperheight}}{mkstapylton@gmail.com-mattS-qiime2_files/mkstapylton@gmail.com-mattS-qiime2_18_0.png}
    \end{center}
    { \hspace*{\fill} \\}
    
\vspace{5mm}%
{\raggedright{}%
    \texttt{Legend grouped by Housing}\\
    \texttt{Color\hspace{3mm}Abundance\hspace{3mm}OTU} \\
    \vspace{3mm}%
    \foreach \A / \B in \otuPhylumHousing {
        \hspace{1mm}\crule[\A]{5mm}{5mm}\hspace{5mm} \texttt{\B\%\hspace{8mm}\A}\\
    }
}%
\vspace{5mm}%
    The above plot represents the percentage of each sample belonging to
particular taxon summarized at the Phylum level.

    \pagebreak

    
    \begin{center}
    \adjustimage{max size={0.9\linewidth}{0.9\paperheight}}{mkstapylton@gmail.com-mattS-qiime2_files/mkstapylton@gmail.com-mattS-qiime2_25_0.png}
    \end{center}
    { \hspace*{\fill} \\}
    
\vspace{5mm}%
{\raggedright{}%
    \texttt{Legend grouped by Strain}\\
    \texttt{Color\hspace{3mm}Abundance\hspace{3mm}OTU} \\
    \vspace{3mm}%
    \foreach \A / \B in \otuPhylumStrain {
        \hspace{1mm}\crule[\A]{5mm}{5mm}\hspace{5mm} \texttt{\B\%\hspace{8mm}\A}\\
    }
}%
\vspace{5mm}%
    The above plot represents the percentage of each sample belonging to
particular taxon summarized at the Phylum level.

    \pagebreak

    \hypertarget{genus-level}{%
\subsection{Genus Level}\label{genus-level}}

    
    \begin{center}
    \adjustimage{max size={0.9\linewidth}{0.9\paperheight}}{mkstapylton@gmail.com-mattS-qiime2_files/mkstapylton@gmail.com-mattS-qiime2_34_0.png}
    \end{center}
    { \hspace*{\fill} \\}
    
\vspace{5mm}%
{\raggedright{}%
    \texttt{Legend grouped by Housing}\\
    \texttt{Color\hspace{3mm}Abundance\hspace{3mm}OTU} \\
    \vspace{3mm}%
    \foreach \A / \B in \otuGenusHousing {
        \hspace{1mm}\crule[\A]{5mm}{5mm}\hspace{5mm} \texttt{\B\%\hspace{8mm}\A}\\
    }
}%
\vspace{5mm}%
    The above plot represents the percentage of each sample belonging to
particular taxon summarized at the Genus level.

    \pagebreak

    
    \begin{center}
    \adjustimage{max size={0.9\linewidth}{0.9\paperheight}}{mkstapylton@gmail.com-mattS-qiime2_files/mkstapylton@gmail.com-mattS-qiime2_41_0.png}
    \end{center}
    { \hspace*{\fill} \\}
    
\vspace{5mm}%
{\raggedright{}%
    \texttt{Legend grouped by Strain}\\
    \texttt{Color\hspace{3mm}Abundance\hspace{3mm}OTU} \\
    \vspace{3mm}%
    \foreach \A / \B in \otuGenusStrain {
        \hspace{1mm}\crule[\A]{5mm}{5mm}\hspace{5mm} \texttt{\B\%\hspace{8mm}\A}\\
    }
}%
\vspace{5mm}%
    The above plot represents the percentage of each sample belonging to
particular taxon summarized at the Genus level.

    \pagebreak

    \hypertarget{alpha-diversity}{%
\section{Alpha Diversity}\label{alpha-diversity}}

    \hypertarget{interpreting-alpha-diversity-results}{%
\subsection{Interpreting Alpha Diversity
Results}\label{interpreting-alpha-diversity-results}}

    Alpha diversity estimates the amount of microbial diversity present in a
sample or group of samples. There are several measures that can be used
for alpha diversity, including observed features, Shannon's diversity or
Faith's phylogenetic diversity. Because diversity estimates depend on
the total number of sequences assigned to each sample, rarefaction
curves are constructed to show the relation between alpha diversity (on
the vertical axis) and sequencing depth (on the horizontal axis). Curves
that gradually plateau as sequencing depth increases suggest that
additional sequencing effort would not substantially yield additional
results in terms of currently not observed diversity; curves that
continue to increase suggest additional sequencing effort might be
required to saturate the estimate.

    \hypertarget{shannon-diversity}{%
\subsection{Shannon Diversity}\label{shannon-diversity}}

    
    \begin{center}
    \adjustimage{max size={0.9\linewidth}{0.9\paperheight}}{mkstapylton@gmail.com-mattS-qiime2_files/mkstapylton@gmail.com-mattS-qiime2_52_0.png}
    \end{center}
    { \hspace*{\fill} \\}
    
\vspace{5mm}%
{\raggedright{}%
    \texttt{Legend for Housing}\\
    \texttt{Color\hspace{3mm}Metadata}\\
    \vspace{3mm}%
    \foreach \A / \B in \Housing {
        \hspace{1mm}\crule[\A]{5mm}{5mm}\hspace{7mm}\texttt{\B}\\%
    }
}%
\vspace{5mm}%\vspace{5mm}%
{\raggedright{}%
    \texttt{Legend for Strain}\\
    \texttt{Color\hspace{3mm}Metadata}\\
    \vspace{3mm}%
    \foreach \A / \B in \Strain {
        \hspace{1mm}\crule[\A]{5mm}{5mm}\hspace{7mm}\texttt{\B}\\%
    }
}%
\vspace{5mm}%
    The above plot represents the average value of alpha diversity at each
sampling depth. The error bars show the standard error within each
group. Groups are determined by the metadata value in each category
specified in the plot.

    \pagebreak

    \hypertarget{faiths-phylogenetic-diversity}{%
\subsection{Faith's Phylogenetic
Diversity}\label{faiths-phylogenetic-diversity}}

    
    \begin{center}
    \adjustimage{max size={0.9\linewidth}{0.9\paperheight}}{mkstapylton@gmail.com-mattS-qiime2_files/mkstapylton@gmail.com-mattS-qiime2_58_0.png}
    \end{center}
    { \hspace*{\fill} \\}
    
\vspace{5mm}%
{\raggedright{}%
    \texttt{Legend for Housing}\\
    \texttt{Color\hspace{3mm}Metadata}\\
    \vspace{3mm}%
    \foreach \A / \B in \Housing {
        \hspace{1mm}\crule[\A]{5mm}{5mm}\hspace{7mm}\texttt{\B}\\%
    }
}%
\vspace{5mm}%\vspace{5mm}%
{\raggedright{}%
    \texttt{Legend for Strain}\\
    \texttt{Color\hspace{3mm}Metadata}\\
    \vspace{3mm}%
    \foreach \A / \B in \Strain {
        \hspace{1mm}\crule[\A]{5mm}{5mm}\hspace{7mm}\texttt{\B}\\%
    }
}%
\vspace{5mm}%
    The above plot represents the average value of alpha diversity at each
sampling depth. The error bars show the standard error within each
group. Groups are determined by the metadata value in each category
specified in the plot.

    \pagebreak

    \hypertarget{observed-asv}{%
\subsection{Observed ASV}\label{observed-asv}}

    
    \begin{center}
    \adjustimage{max size={0.9\linewidth}{0.9\paperheight}}{mkstapylton@gmail.com-mattS-qiime2_files/mkstapylton@gmail.com-mattS-qiime2_64_0.png}
    \end{center}
    { \hspace*{\fill} \\}
    
\vspace{5mm}%
{\raggedright{}%
    \texttt{Legend for Housing}\\
    \texttt{Color\hspace{3mm}Metadata}\\
    \vspace{3mm}%
    \foreach \A / \B in \Housing {
        \hspace{1mm}\crule[\A]{5mm}{5mm}\hspace{7mm}\texttt{\B}\\%
    }
}%
\vspace{5mm}%\vspace{5mm}%
{\raggedright{}%
    \texttt{Legend for Strain}\\
    \texttt{Color\hspace{3mm}Metadata}\\
    \vspace{3mm}%
    \foreach \A / \B in \Strain {
        \hspace{1mm}\crule[\A]{5mm}{5mm}\hspace{7mm}\texttt{\B}\\%
    }
}%
\vspace{5mm}%
    The above plot represents the average value of alpha diversity at each
sampling depth. The error bars show the standard error within each
group. Groups are determined by the metadata value in each category
specified in the plot.

    \pagebreak

    \hypertarget{beta-diversity}{%
\section{Beta Diversity}\label{beta-diversity}}

    \hypertarget{interpreting-beta-diversity-results}{%
\subsection{Interpreting Beta Diversity
Results}\label{interpreting-beta-diversity-results}}

    Beta diversity estimates how similar or dissimilar samples are based on
their microbiome composition. Different to alpha diversity, which is
estimated per sample, beta diversity is a distance that is calculated
between pairs of samples. Samples that are similar to each other in
their microbiome composition will have a low distance between them based
on beta diversity, while those that are very different in their
composition will have a large distance. Principal Coordinate Analysis
(PCoA) is an ordination technique that visually represents the samples
based on their beta diversity distances to facilitate the identification
of clusters or gradients of samples. By default, the first three
principal coordinates are shown in PCoA plots.

    \hypertarget{bray-curtis-grouped-by-housing}{%
\subsection{Bray-Curtis, grouped by
Housing}\label{bray-curtis-grouped-by-housing}}

    
    \begin{center}
    \adjustimage{max size={0.9\linewidth}{0.9\paperheight}}{mkstapylton@gmail.com-mattS-qiime2_files/mkstapylton@gmail.com-mattS-qiime2_74_0.png}
    \end{center}
    { \hspace*{\fill} \\}
    
\vspace{5mm}%
{\raggedright{}%
    \texttt{Legend for Housing}\\
    \texttt{Color\hspace{3mm}Metadata}\\
    \vspace{3mm}%
    \foreach \A / \B in \Housing {
        \hspace{1mm}\crule[\A]{5mm}{5mm}\hspace{7mm}\texttt{\B}\\%
    }
}%
\vspace{5mm}%
    The above plot represents the first three compenents created when
performing Principle Component Analysis on the Beta diversity of the
samples.

    
    \begin{center}
    \adjustimage{max size={0.9\linewidth}{0.9\paperheight}}{mkstapylton@gmail.com-mattS-qiime2_files/mkstapylton@gmail.com-mattS-qiime2_76_0.png}
    \end{center}
    { \hspace*{\fill} \\}
    
\vspace{5mm}%
{\raggedright{}%
    \texttt{Legend for Housing}\\
    \texttt{Color\hspace{3mm}Metadata}\\
    \vspace{3mm}%
    \foreach \A / \B in \Housing {
        \hspace{1mm}\crule[\A]{5mm}{5mm}\hspace{7mm}\texttt{\B}\\%
    }
}%
\vspace{5mm}%
    
    \begin{center}
    \adjustimage{max size={0.9\linewidth}{0.9\paperheight}}{mkstapylton@gmail.com-mattS-qiime2_files/mkstapylton@gmail.com-mattS-qiime2_77_0.png}
    \end{center}
    { \hspace*{\fill} \\}
    
\vspace{5mm}%
{\raggedright{}%
    \texttt{Legend for Housing}\\
    \texttt{Color\hspace{3mm}Metadata}\\
    \vspace{3mm}%
    \foreach \A / \B in \Housing {
        \hspace{1mm}\crule[\A]{5mm}{5mm}\hspace{7mm}\texttt{\B}\\%
    }
}%
\vspace{5mm}%
    
    \begin{center}
    \adjustimage{max size={0.9\linewidth}{0.9\paperheight}}{mkstapylton@gmail.com-mattS-qiime2_files/mkstapylton@gmail.com-mattS-qiime2_78_0.png}
    \end{center}
    { \hspace*{\fill} \\}
    
\vspace{5mm}%
{\raggedright{}%
    \texttt{Legend for Housing}\\
    \texttt{Color\hspace{3mm}Metadata}\\
    \vspace{3mm}%
    \foreach \A / \B in \Housing {
        \hspace{1mm}\crule[\A]{5mm}{5mm}\hspace{7mm}\texttt{\B}\\%
    }
}%
\vspace{5mm}%
    \pagebreak

    \hypertarget{bray-curtis-grouped-by-strain}{%
\subsection{Bray-Curtis, grouped by
Strain}\label{bray-curtis-grouped-by-strain}}

    
    \begin{center}
    \adjustimage{max size={0.9\linewidth}{0.9\paperheight}}{mkstapylton@gmail.com-mattS-qiime2_files/mkstapylton@gmail.com-mattS-qiime2_83_0.png}
    \end{center}
    { \hspace*{\fill} \\}
    
\vspace{5mm}%
{\raggedright{}%
    \texttt{Legend for Strain}\\
    \texttt{Color\hspace{3mm}Metadata}\\
    \vspace{3mm}%
    \foreach \A / \B in \Strain {
        \hspace{1mm}\crule[\A]{5mm}{5mm}\hspace{7mm}\texttt{\B}\\%
    }
}%
\vspace{5mm}%
    The above plot represents the first three compenents created when
performing Principle Component Analysis on the Beta diversity of the
samples.

    
    \begin{center}
    \adjustimage{max size={0.9\linewidth}{0.9\paperheight}}{mkstapylton@gmail.com-mattS-qiime2_files/mkstapylton@gmail.com-mattS-qiime2_85_0.png}
    \end{center}
    { \hspace*{\fill} \\}
    
\vspace{5mm}%
{\raggedright{}%
    \texttt{Legend for Strain}\\
    \texttt{Color\hspace{3mm}Metadata}\\
    \vspace{3mm}%
    \foreach \A / \B in \Strain {
        \hspace{1mm}\crule[\A]{5mm}{5mm}\hspace{7mm}\texttt{\B}\\%
    }
}%
\vspace{5mm}%
    
    \begin{center}
    \adjustimage{max size={0.9\linewidth}{0.9\paperheight}}{mkstapylton@gmail.com-mattS-qiime2_files/mkstapylton@gmail.com-mattS-qiime2_86_0.png}
    \end{center}
    { \hspace*{\fill} \\}
    
\vspace{5mm}%
{\raggedright{}%
    \texttt{Legend for Strain}\\
    \texttt{Color\hspace{3mm}Metadata}\\
    \vspace{3mm}%
    \foreach \A / \B in \Strain {
        \hspace{1mm}\crule[\A]{5mm}{5mm}\hspace{7mm}\texttt{\B}\\%
    }
}%
\vspace{5mm}%
    
    \begin{center}
    \adjustimage{max size={0.9\linewidth}{0.9\paperheight}}{mkstapylton@gmail.com-mattS-qiime2_files/mkstapylton@gmail.com-mattS-qiime2_87_0.png}
    \end{center}
    { \hspace*{\fill} \\}
    
\vspace{5mm}%
{\raggedright{}%
    \texttt{Legend for Strain}\\
    \texttt{Color\hspace{3mm}Metadata}\\
    \vspace{3mm}%
    \foreach \A / \B in \Strain {
        \hspace{1mm}\crule[\A]{5mm}{5mm}\hspace{7mm}\texttt{\B}\\%
    }
}%
\vspace{5mm}%
    \pagebreak

    \hypertarget{unweighted-unifrac-grouped-by-housing}{%
\subsection{Unweighted UniFrac, grouped by
Housing}\label{unweighted-unifrac-grouped-by-housing}}

    
    \begin{center}
    \adjustimage{max size={0.9\linewidth}{0.9\paperheight}}{mkstapylton@gmail.com-mattS-qiime2_files/mkstapylton@gmail.com-mattS-qiime2_92_0.png}
    \end{center}
    { \hspace*{\fill} \\}
    
\vspace{5mm}%
{\raggedright{}%
    \texttt{Legend for Housing}\\
    \texttt{Color\hspace{3mm}Metadata}\\
    \vspace{3mm}%
    \foreach \A / \B in \Housing {
        \hspace{1mm}\crule[\A]{5mm}{5mm}\hspace{7mm}\texttt{\B}\\%
    }
}%
\vspace{5mm}%
    The above plot represents the first three compenents created when
performing Principle Component Analysis on the Beta diversity of the
samples.

    
    \begin{center}
    \adjustimage{max size={0.9\linewidth}{0.9\paperheight}}{mkstapylton@gmail.com-mattS-qiime2_files/mkstapylton@gmail.com-mattS-qiime2_94_0.png}
    \end{center}
    { \hspace*{\fill} \\}
    
\vspace{5mm}%
{\raggedright{}%
    \texttt{Legend for Housing}\\
    \texttt{Color\hspace{3mm}Metadata}\\
    \vspace{3mm}%
    \foreach \A / \B in \Housing {
        \hspace{1mm}\crule[\A]{5mm}{5mm}\hspace{7mm}\texttt{\B}\\%
    }
}%
\vspace{5mm}%
    
    \begin{center}
    \adjustimage{max size={0.9\linewidth}{0.9\paperheight}}{mkstapylton@gmail.com-mattS-qiime2_files/mkstapylton@gmail.com-mattS-qiime2_95_0.png}
    \end{center}
    { \hspace*{\fill} \\}
    
\vspace{5mm}%
{\raggedright{}%
    \texttt{Legend for Housing}\\
    \texttt{Color\hspace{3mm}Metadata}\\
    \vspace{3mm}%
    \foreach \A / \B in \Housing {
        \hspace{1mm}\crule[\A]{5mm}{5mm}\hspace{7mm}\texttt{\B}\\%
    }
}%
\vspace{5mm}%
    
    \begin{center}
    \adjustimage{max size={0.9\linewidth}{0.9\paperheight}}{mkstapylton@gmail.com-mattS-qiime2_files/mkstapylton@gmail.com-mattS-qiime2_96_0.png}
    \end{center}
    { \hspace*{\fill} \\}
    
\vspace{5mm}%
{\raggedright{}%
    \texttt{Legend for Housing}\\
    \texttt{Color\hspace{3mm}Metadata}\\
    \vspace{3mm}%
    \foreach \A / \B in \Housing {
        \hspace{1mm}\crule[\A]{5mm}{5mm}\hspace{7mm}\texttt{\B}\\%
    }
}%
\vspace{5mm}%
    \pagebreak

    \hypertarget{unweighted-unifrac-grouped-by-strain}{%
\subsection{Unweighted UniFrac, grouped by
Strain}\label{unweighted-unifrac-grouped-by-strain}}

    
    \begin{center}
    \adjustimage{max size={0.9\linewidth}{0.9\paperheight}}{mkstapylton@gmail.com-mattS-qiime2_files/mkstapylton@gmail.com-mattS-qiime2_101_0.png}
    \end{center}
    { \hspace*{\fill} \\}
    
\vspace{5mm}%
{\raggedright{}%
    \texttt{Legend for Strain}\\
    \texttt{Color\hspace{3mm}Metadata}\\
    \vspace{3mm}%
    \foreach \A / \B in \Strain {
        \hspace{1mm}\crule[\A]{5mm}{5mm}\hspace{7mm}\texttt{\B}\\%
    }
}%
\vspace{5mm}%
    The above plot represents the first three compenents created when
performing Principle Component Analysis on the Beta diversity of the
samples.

    
    \begin{center}
    \adjustimage{max size={0.9\linewidth}{0.9\paperheight}}{mkstapylton@gmail.com-mattS-qiime2_files/mkstapylton@gmail.com-mattS-qiime2_103_0.png}
    \end{center}
    { \hspace*{\fill} \\}
    
\vspace{5mm}%
{\raggedright{}%
    \texttt{Legend for Strain}\\
    \texttt{Color\hspace{3mm}Metadata}\\
    \vspace{3mm}%
    \foreach \A / \B in \Strain {
        \hspace{1mm}\crule[\A]{5mm}{5mm}\hspace{7mm}\texttt{\B}\\%
    }
}%
\vspace{5mm}%
    
    \begin{center}
    \adjustimage{max size={0.9\linewidth}{0.9\paperheight}}{mkstapylton@gmail.com-mattS-qiime2_files/mkstapylton@gmail.com-mattS-qiime2_104_0.png}
    \end{center}
    { \hspace*{\fill} \\}
    
\vspace{5mm}%
{\raggedright{}%
    \texttt{Legend for Strain}\\
    \texttt{Color\hspace{3mm}Metadata}\\
    \vspace{3mm}%
    \foreach \A / \B in \Strain {
        \hspace{1mm}\crule[\A]{5mm}{5mm}\hspace{7mm}\texttt{\B}\\%
    }
}%
\vspace{5mm}%
    
    \begin{center}
    \adjustimage{max size={0.9\linewidth}{0.9\paperheight}}{mkstapylton@gmail.com-mattS-qiime2_files/mkstapylton@gmail.com-mattS-qiime2_105_0.png}
    \end{center}
    { \hspace*{\fill} \\}
    
\vspace{5mm}%
{\raggedright{}%
    \texttt{Legend for Strain}\\
    \texttt{Color\hspace{3mm}Metadata}\\
    \vspace{3mm}%
    \foreach \A / \B in \Strain {
        \hspace{1mm}\crule[\A]{5mm}{5mm}\hspace{7mm}\texttt{\B}\\%
    }
}%
\vspace{5mm}%
    \pagebreak

    \hypertarget{weighted-unifrac-grouped-by-housing}{%
\subsection{Weighted UniFrac, grouped by
Housing}\label{weighted-unifrac-grouped-by-housing}}

    
    \begin{center}
    \adjustimage{max size={0.9\linewidth}{0.9\paperheight}}{mkstapylton@gmail.com-mattS-qiime2_files/mkstapylton@gmail.com-mattS-qiime2_110_0.png}
    \end{center}
    { \hspace*{\fill} \\}
    
\vspace{5mm}%
{\raggedright{}%
    \texttt{Legend for Housing}\\
    \texttt{Color\hspace{3mm}Metadata}\\
    \vspace{3mm}%
    \foreach \A / \B in \Housing {
        \hspace{1mm}\crule[\A]{5mm}{5mm}\hspace{7mm}\texttt{\B}\\%
    }
}%
\vspace{5mm}%
    The above plot represents the first three compenents created when
performing Principle Component Analysis on the Beta diversity of the
samples.

    
    \begin{center}
    \adjustimage{max size={0.9\linewidth}{0.9\paperheight}}{mkstapylton@gmail.com-mattS-qiime2_files/mkstapylton@gmail.com-mattS-qiime2_112_0.png}
    \end{center}
    { \hspace*{\fill} \\}
    
\vspace{5mm}%
{\raggedright{}%
    \texttt{Legend for Housing}\\
    \texttt{Color\hspace{3mm}Metadata}\\
    \vspace{3mm}%
    \foreach \A / \B in \Housing {
        \hspace{1mm}\crule[\A]{5mm}{5mm}\hspace{7mm}\texttt{\B}\\%
    }
}%
\vspace{5mm}%
    
    \begin{center}
    \adjustimage{max size={0.9\linewidth}{0.9\paperheight}}{mkstapylton@gmail.com-mattS-qiime2_files/mkstapylton@gmail.com-mattS-qiime2_113_0.png}
    \end{center}
    { \hspace*{\fill} \\}
    
\vspace{5mm}%
{\raggedright{}%
    \texttt{Legend for Housing}\\
    \texttt{Color\hspace{3mm}Metadata}\\
    \vspace{3mm}%
    \foreach \A / \B in \Housing {
        \hspace{1mm}\crule[\A]{5mm}{5mm}\hspace{7mm}\texttt{\B}\\%
    }
}%
\vspace{5mm}%
    
    \begin{center}
    \adjustimage{max size={0.9\linewidth}{0.9\paperheight}}{mkstapylton@gmail.com-mattS-qiime2_files/mkstapylton@gmail.com-mattS-qiime2_114_0.png}
    \end{center}
    { \hspace*{\fill} \\}
    
\vspace{5mm}%
{\raggedright{}%
    \texttt{Legend for Housing}\\
    \texttt{Color\hspace{3mm}Metadata}\\
    \vspace{3mm}%
    \foreach \A / \B in \Housing {
        \hspace{1mm}\crule[\A]{5mm}{5mm}\hspace{7mm}\texttt{\B}\\%
    }
}%
\vspace{5mm}%
    \pagebreak

    \hypertarget{weighted-unifrac-grouped-by-strain}{%
\subsection{Weighted UniFrac, grouped by
Strain}\label{weighted-unifrac-grouped-by-strain}}

    
    \begin{center}
    \adjustimage{max size={0.9\linewidth}{0.9\paperheight}}{mkstapylton@gmail.com-mattS-qiime2_files/mkstapylton@gmail.com-mattS-qiime2_119_0.png}
    \end{center}
    { \hspace*{\fill} \\}
    
\vspace{5mm}%
{\raggedright{}%
    \texttt{Legend for Strain}\\
    \texttt{Color\hspace{3mm}Metadata}\\
    \vspace{3mm}%
    \foreach \A / \B in \Strain {
        \hspace{1mm}\crule[\A]{5mm}{5mm}\hspace{7mm}\texttt{\B}\\%
    }
}%
\vspace{5mm}%
    The above plot represents the first three compenents created when
performing Principle Component Analysis on the Beta diversity of the
samples.

    
    \begin{center}
    \adjustimage{max size={0.9\linewidth}{0.9\paperheight}}{mkstapylton@gmail.com-mattS-qiime2_files/mkstapylton@gmail.com-mattS-qiime2_121_0.png}
    \end{center}
    { \hspace*{\fill} \\}
    
\vspace{5mm}%
{\raggedright{}%
    \texttt{Legend for Strain}\\
    \texttt{Color\hspace{3mm}Metadata}\\
    \vspace{3mm}%
    \foreach \A / \B in \Strain {
        \hspace{1mm}\crule[\A]{5mm}{5mm}\hspace{7mm}\texttt{\B}\\%
    }
}%
\vspace{5mm}%
    
    \begin{center}
    \adjustimage{max size={0.9\linewidth}{0.9\paperheight}}{mkstapylton@gmail.com-mattS-qiime2_files/mkstapylton@gmail.com-mattS-qiime2_122_0.png}
    \end{center}
    { \hspace*{\fill} \\}
    
\vspace{5mm}%
{\raggedright{}%
    \texttt{Legend for Strain}\\
    \texttt{Color\hspace{3mm}Metadata}\\
    \vspace{3mm}%
    \foreach \A / \B in \Strain {
        \hspace{1mm}\crule[\A]{5mm}{5mm}\hspace{7mm}\texttt{\B}\\%
    }
}%
\vspace{5mm}%
    
    \begin{center}
    \adjustimage{max size={0.9\linewidth}{0.9\paperheight}}{mkstapylton@gmail.com-mattS-qiime2_files/mkstapylton@gmail.com-mattS-qiime2_123_0.png}
    \end{center}
    { \hspace*{\fill} \\}
    
\vspace{5mm}%
{\raggedright{}%
    \texttt{Legend for Strain}\\
    \texttt{Color\hspace{3mm}Metadata}\\
    \vspace{3mm}%
    \foreach \A / \B in \Strain {
        \hspace{1mm}\crule[\A]{5mm}{5mm}\hspace{7mm}\texttt{\B}\\%
    }
}%
\vspace{5mm}%
    \pagebreak


    % Add a bibliography block to the postdoc
    
    
\bibliography{references}

    
\end{document}
